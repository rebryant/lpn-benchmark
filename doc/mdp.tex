\documentclass[10pt,conference]{IEEEtran}


\usepackage{bm}
%% Colored hyperlink 
\usepackage{cite}
\usepackage{hyperref}
\newcommand{\cref}[2]{\href{#1}{#2}}
%% Colored hyperlink showing link in TT font
% \newcommand{\chref}[1]{\href{#1}{\small\tt \color{blue}#1}}
\newcommand{\hcref}[1]{\cref{#1}{\small\tt #1}}
\newcommand{\ints}[1]{{\cal N}_{#1}}
\newcommand{\bset}{{\cal B}}
\newcommand{\kissat}{{\sffamily\scshape kissat}}
\newcommand{\Kissat}{{\sffamily\scshape  Kissat}}
\newcommand{\cms}{{\sffamily\scshape  CryptoMiniSat}}
\newcommand{\Cms}{{\sffamily\scshape  CryptoMiniSat}}


\bibliographystyle{IEEEtran}

\begin{document}

\title{Minimum Disagreement Parity (MDP) Benchmark\thanks{This work was supported by the U. S. National Science Foundation under grant CCF-2108521}}

\author{\IEEEauthorblockN{Randal E. Bryant}
\IEEEauthorblockA{Computer Science Department \\
Carnegie Mellon University, Pittsburgh, PA, United States\\
Email: {\tt Randy.Bryant@cs.cmu.edu}}}

\maketitle

\section{Obtaining Benchmarks}

The benchmark generator, files, and documentation are available at:
\begin{center}
\hcref{https://github.com/rebryant/mdp-benchmark}
\end{center}

\section{Description}

Crawford, Kearns, and Shapire proposed the Minimum Disagreement Parity (MDP)
Problem as a challenging SAT benchmark in an unpublished report from
AT\&T Bell Laboratories in 1994~\cite{crawford:mdp:1994}.

MDP is closely related to the ``Learning Parity with Noise'' (LPN)
problem.  LPN has been proposed as the basis for public key
crytographic systems~\cite{katz:crypto:2007}.  Unlike the widely used
RSA cryptosystem, it is resistant to all known quantum
algorithms~\cite{pietrzak:sofsem:2012}.  The capabilities of SAT
solvers on MDP is therefore of interest to the cryptography community.
We provide these benchmarks as a way to stimulate 
the SAT community to expand beyond pure CDCL, incorporating other
solution methods into their SAT solvers.

Crawford wrote a benchmark generator in the 1990s and supplied several
files to early SAT competitions with names of the form
\texttt{par}$N$\texttt{-}$S$\texttt{.cnf}, where $N$ is the size
parameter and $S$ is a random seed.
These had values of $N \in \{8, 16, 32\}$.  These files are still available online at \hcref{https://www.cs.ubc.ca/~hoos/SATLIB/benchm.html}.
SAT solvers of that era were challenged by $N=16$ and
could not possibly handle $N=32$.  Unfortunately, the code for his
benchmark generator has disappeared.

We wrote a new benchmark generator for the MDP problem.  In doing so, we
added more options for problem parameters and encoding methods.  We
also replaced the binary encoding of at-most-$k$ constraints with a
more SAT-solver-friendly unary counter encoding~\cite{card}.

\section{Problem Description}

In the following, let $\bset = \{0, 1\}$ and $\ints{p} = \{1, 2, \ldots, p\}$.
Assume all arithmetic is performed modulo 2.  Thus,
if $a, b \in \bset$, then $a+b \equiv a \oplus b$.

The problem is parameterized by a number of solution bits $n$, a
number of samples $m$, and an error tolerance $k$, as follows.  Let
$\bm{s} = s_1, s_2, \ldots, s_n$ be a set of {\em solution} bits.
For $1 \leq j \leq m$, let $X_j \subseteq \ints{n}$
be a {\em sample set}, created by generating $n$
random bits $x_{1,j}, x_{2,j}, \ldots x_{n,j}$
and letting $X_j = \{i | x_{i,j} = 1 \}$.  Let $\bm{y} = y_1, y_2, \ldots, y_m$ be the
parities of the solution bits for each of the $m$ samples:
\begin{eqnarray}
y_j & = & \sum_{i \in X_j} s_i \label{eqn:uncorrupted}
\end{eqnarray}

Given enough many samples $m$ for there to be at least $n$ linearly
independent sample sets, the values of the solution bits $\bm{s}$ can
be uniquely determined from $\bm{y}$ and the sample sets $S_j$ for
$1 \leq j \leq m$ by Gaussian elimination.  To make this problem
challenging, we introduce ``noise,'' allowing up to $k$ of these
samples to be ``corrupted'' by flipping the values of their parity.
That is, let $T \subseteq \ints{m}$ be created by randomly choosing
$k$ values from $\ints{m}$ without replacement, and define $m$
``corruption'' bits $\bm{r} = r_1, r_2, \ldots, r_m$, with $r_j$ equal
to $1$ if $j \in T$ and equal to $0$ otherwise.  
We then provide noisy samples $\bm{y}'$, defined as:
\begin{eqnarray}
y'_j & = & r_j + \sum_{i \in X_j} s_i \label{eqn:corrupted}
\end{eqnarray}
and require the correct solution bits $\bm{s}$ to be determined despite this noise.  That is, the generated solution $\bm{s}$ must satisfy at least $m-k$ of equations (\ref{eqn:uncorrupted}).
For larger values of $k$, the
problem becomes NP-hard.

This problem can readily be encoded in CNF with variables for
unknown values $\bm{s}$ and $\bm{r}$, along with some auxilliary
variables.  We further parameterize the problem with a value
$t \leq k$, indicating the maximum number of corrupted samples accepted in the
solution, where the problem should be satisfiable when $t = k$ but may
become unsatisfiable for $t = k-1$.  Each of the $m$ equations
(\ref{eqn:corrupted}) is encoded using auxilliary variables to avoid
exponential expansion.  An at-most-$t$ constraint is imposed on the
corruption bits $\bm{r}$.

For $t=k$, the solution $\bm{s}$ is not guaranteed to be unique, but
we allow any solution that satisfies at least $m-k$ of the constraints
(\ref{eqn:uncorrupted}).  In addition, setting $t=k-1$ does not guarantee that the
formula is unsatisfiable.  Indeed, we found some instances where there was a solution that satisfied $m-k+1$ constraints.

By analyzing the CNF file, it is
possible to discern the sample sets $S_j$ and the values of the noisy
samples $\bm{y}'$.  The values of $\bm{s}$ and $\bm{r}$, however,
remain hidden, except as comments at the start of the file.

Crawford suggests choosing $n$ to be a multiple of 4 and letting
$m = 2n$ and $k = m/8 = n/4$.  Our benchmark files were all generated under
those conditions.

\section{Provided files}

The generator \texttt{mdp-gen.py} and an associated README file are
located in the \texttt{src} subdirectory.  This directory also
contains a program \texttt{mdp-check.py}.  Given a \texttt{.cnf} file
generated by \texttt{mdp-gen.py} and the output of a successful run of
a SAT solver, it can check that the solution for the input variables
representing $\bm{s}$ indeed satisfies at least $m-k$ of the equations of (\ref{eqn:uncorrupted}).
The supplied benchmark files were generated by running the script
\texttt{generate.sh} in the \texttt{src} subdirectory.

There are 30 files: five satisfiable and five unsatisfiable instances
for $n \in \{28, 32, 36\}$, generated using five different random
seeds.  The seeds were adjusted so that the 15 instances generated
with $t=k-1$ are all unsatisfiable, as is discussed below.

We tested these benchmarks using the
\kissat{}~\cite{biere-kissat-2020} CDCL solver.  Measurements were
peformed on a 3.2~GHz Apple M1~Max processor with 64~GB of memory and
running the OS~X operating system, with a time limit of 5000 seconds
per run.  For $n=28$, \Kissat{} can easily handle both the satisfiable
and the unsatisfiable instances, with times ranging from
30 to 900 seconds.  For the satisfiable instances with $n=32$, it can
solve some in just 30 seconds, but times out for two of the five
runs.  It times out on all unsatisfiable instances for $n=32$, and
it times out on all satisfiable and unsatisfiable instances for
$n=36$.

We also tested the benchmarks with \cms{}, a CDCL solver augmented
with the ability to perform Gauss-Jordan elimination on parity
constraints~\cite{crypto}.  It can easily handle all satisfiable
instances, never requiring more than 90 seconds.  When not required to
generate a proof of unsatisfiability, it can also easily handle all of
the unsatisfiable instances.  That is how we ensured that the instances
with $t=k-1$ are unsatisfiable.
Currently, \cms{} cannot generate DRAT proofs of
unsatisfiability when it uses Gauss-Jordan elimination, 
and so it fares no better than \kissat{} on the
unsatisfiable instances when proof generation is required.

\newpage

\Cms{} can scale to $n=60$ without
difficulty.  Nonetheless, the problem is still NP-hard, and so even
\cms{} only pushes the boundary before exponential scaling limits
feasibility.  

\bibliography{mdp}
\end{document}
